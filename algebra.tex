% Created 2024-04-30 Tue 21:29
% Intended LaTeX compiler: pdflatex
\documentclass[11pt]{article}
\usepackage[utf8]{inputenc}
\usepackage[T1]{fontenc}
\usepackage{graphicx}
\usepackage{grffile}
\usepackage{longtable}
\usepackage{wrapfig}
\usepackage{rotating}
\usepackage[normalem]{ulem}
\usepackage{amsmath}
\usepackage{textcomp}
\usepackage{amssymb}
\usepackage{capt-of}
\usepackage{hyperref}
\setcounter{secnumdepth}{2}
\author{ychnh}
\date{\today}
\title{}
\hypersetup{
 pdfauthor={ychnh},
 pdftitle={},
 pdfkeywords={},
 pdfsubject={},
 pdfcreator={Emacs 27.1 (Org mode 9.3)}, 
 pdflang={English}}
\begin{document}



\section{Quotes}
\label{sec:org6c1edcb}
\subsection{If it is true, how can you get there?}
\label{sec:orgbe2f326}
\subsection{First aim for consistency. Then try for speed. - CPS on TOA}
\label{sec:orgd167be9}
\subsection{Let's see, is there a just a numerical reason why that would be true? - Kats}
\label{sec:orgfc57ed0}
on \(n|m \iff p^n-1 | p^m-1\)
\subsection{Lets see, why is it so compliated? \(n = \sum_{d|n} I_d(x)\). Can you use the inherent properties of multiplication (groups) to show it is true? These guys are the total collection of elements that are the elements of order x st x|n in \(G=Z_n\)}
\label{sec:orgdd36cd8}
\subsection{See how it fails and why it fails, it should suggest how to correct it.}
\label{sec:org50238ce}

\section{Galois}
\label{sec:org6140f42}
\subsection{\textbf{PROP} \(GCD(f(x), f'(x) )=1 \iff\) \(f(x), f'(x)\) have no roots in common \(\iff\) f(x) has distinct roots in its splitting field}
\label{sec:orgcd95308}
\subsection{\textbf{COR} \(f(x)\) irreducible \(\implies\) \(f(x)\) has distinct roots in its splitting field}
\label{sec:org7bebe0d}
\subsection{\textbf{DEF} Let \(F \subseteq K\). \(\alpha \in K\) is \textbf{algebraic} over F if there is \(f(x)\in F[x]\) s.t. \(f(\alpha)=0\)}
\label{sec:org7a4899c}
\subsection{\textbf{RMK} Determinant trick: for \(\alpha \in K\) gives a polynomial over the base field st \(f(\alpha)=0\)}
\label{sec:orgb74d53c}
\subsection{\textbf{DEF} Let \(F \subseteq K\). K is an \textbf{algebraic extension} if all elements of K are algebraic over F}
\label{sec:org8764f90}
\subsection{\textbf{PROP} Let \(F \subseteq K\). \(\alpha \in K\) is algebraic iff \([F(\alpha):F] < \infty\)}
\label{sec:org5bc486f}
<= If the degree of simple extension is finite, can use determinant trick.
=> If \(\alpha\) is algebraic, there is a minimal poly in F. So the extension to \(F(\alpha)\) is finite.
Intuively it says that finite degree extensions are by their nature, algebraic objects. And the reason closely tied to determinants.
\subsection{\textbf{COR} Finite degree extension is algebraic}
\label{sec:org30c6c3c}
Let \(F \subseteq K\) Take any \(\gamma \in K\). Consider \(F(\gamma) \subseteq K\). It is finite degree extension, so it is algebraic. 
\subsection{\textbf{PROP} Sufficient condition for simple extension.}
\label{sec:org5a3fde3}
Let \(Q \subseteq F\) or \(|F| < \infty\). If the degree of the extension is finite, it simple
ETS show that \(F \subseteq F(u,v)\) is simple.
Case A B
A \(Q \subseteq F\) in order for \(u+\lambda v\) to not be primitive, \(\lambda\) must satistfy conditions dependent on roots of minimal poly for \(u\) and \(v\). Since F infinite, this is not possible.
B \(F\) finite. The \(K\) is cyclic due to gp theory fact. (If \(x^n=1\) has at most n soln. for all n, the G is cyclic. Check Kat's Notes)
\subsection{\textbf{QUES} Find an example of a finite extension that is not simple}
\label{sec:orgc2a51bb}
\subsection{Sums and product of algebraic elements are algebraic}
\label{sec:org77fac7f}
\subsection{Transitivity of Algebraic}
\label{sec:orgb4e1113}
\(F \subseteq E \subseteq K\). K algebraic over E and E algebraic over F then K algebraic over F
\subsection{Construction of Algebraic Closure}
\label{sec:org4d04e23}
Let \(Q \subseteq F\) or \(|F|<\infty\). If \(|K:F|<\infty\) then \(K=F(\alpha)\)
\subsection{\textbf{PROP} Characterization of finitely many intermediate fields (4-8)}
\label{sec:org4eabb3b}
Let \(F \subseteq K\) with \(|F|=\infty\) \([K:F]<\infty\)
Then the extension is simple iff there are only finitely many intermediate fields \(F \subset E \subset K\)
-> if the extension is simple, there is a minimal poly of \(\alpha\) p(x) over F.
Take any intermediate field \(E\) and consider min poly of \(\alpha\) over E
<- ETS for \(F \subseteq F(u,v)\) has finitely many intermediate fields.
\subsection{\textbf{Remark.}}
\label{sec:orgd9d0472}
An automorphism fixing F takes root a \(f(x)\in F[x]\) to another root.
In a primitive field extension, the behavior of \(\alpha\) completely describes the behavior of F
An homomorphism describes the structure between two algebraic sets
An isomorphism says the structure is the same.
If an isomorphism maps generators of one 
Let \(F \subseteq K_1\) \(F \subseteq K_2\). If \(K_1\) is completely described by roots of a single polynomial, and 

\subsection{\textbf{Crucial Prop} extension of base field isomorphism to a simple field extension isomorphism}
\label{sec:orgce4c1ba}
Let \(\sigma : F_1 \rightarrow F_2\) an isomorphism and \(p_1(x)\) min poly of \(\alpha_1\). Let \(p_2(x):=p_1(x)^\sigma\), min poly of \(\alpha_2\). Then we can extend to an isomorphism \(\overline\sigma: F_1(\alpha_1) \rightarrow F_2(\alpha_2)\)
A special case is that a field extension of any element is identical
\subsection{\textbf{COR} Let K be splitting field. If a root of an irreducible poly is in K, then all the roots are in K.}
\label{sec:org5afed12}
Let K be splitting field for f(x). If p(x) is an irreducible polynomial that has a root in K, then all the roots of p(x) are in K.
The proof is very interesting. 

\subsection{\textbf{DEF} Gal(K\F) is called \textbf{Galois} if |Gal(K\F)| = [K:F]}
\label{sec:org8d32724}
\subsection{\textbf{Characterization of Galois.} Let \(K=F(\alpha)\), p(x) deg d min poly of \(\alpha\) over F. Gal(K\F) is Galois iff p(x) has d distinct roots in K.}
\label{sec:org3bbb12d}
Intuition: Because roots of p(x) go to roots under a \(\sigma \in Gal(K/F)\), you need the full set of automorphisms
Conversely, the distinct roots give rise to the full set of automoprhisms
(Example) of when it fails and how it fails, \(\mathbb{Z}_2\) consider \(x^2-1\).

\subsection{TFAE: Let \(Q \subset F\). Then TFAE (a) K is Galois over F (b) K is splitting field of p(x) over F. (c) K is splitting field of some \(f(x)\in F[x]\) over}
\label{sec:org99768f8}
\subsection{\textbf{When is Finite Field Extension Galois.}}
\label{sec:org4f87351}
If \(|F|<\infty\) (Char(F)=p) (\(|K:F| < \infty\) then K is Galois over F 
Since \(K=F(\alpha)\), use the characterization fo Galois. Show that p(x), the minimal poly for \(\alpha\) 

\subsection{\textbf{Definition.} Fixed field of an automorphism or a collection of automorphism.}
\label{sec:org0b81004}
\(K^\sigma := \{k | \sigma(k)=k\}\) \(K^H := \{k | \sigma(k)=k, \forall \sigma \in H \}\)
\subsection{Galois Correspondence Thm.}
\label{sec:orgbd3f4a1}
\subsubsection*{Let \(F \subseteq K\) be finite galois extention.}
\label{sec:orgbb8e6bf}
\subsubsection*{There is a 1-1 correspondence btw \(H \subseteq Gal(K/F)\) and intermediate fields \(F \subseteq E \subseteq K\)}
\label{sec:orgbc0e02a}
The correspondnce is given by \(H \rightarrow K^H \rightarrow Gal(K/K^H)=H\)
?: I understand H is contained in Gal(K/K\textsuperscript{H}), since the maps in H fix K\textsuperscript{H}. But why can't it be more?
The correspondence is given by \(E \rightarrow Gal(K/E) \rightarrow K^{Gal(K/E)} = E\)
?: I understand that E is contained in \(K^{Gal(K/E)}\) since the maps in Gal(K/E) already fix E but why can't it be more?
\subsubsection*{If \(H \leftrightarrow E\) corresond, then [G:H]=[E:F]}
\label{sec:org7f41fa3}
\subsubsection*{K is Galois over any intermediate field E}
\label{sec:org9efcd34}
\subsubsection*{E Galois over F iff Gal(K/E) is normal in Gal(K/F) in which case \(Gal(E/F) \cong \dfrac{Gal(K/F)}{Gal(K/E)}\)}
\label{sec:org8766d97}
\end{document}
